%
% File main.tex
%
% Contact: car@ir.hit.edu.cn, gdzhou@suda.edu.cn
%%e.agirre@ehu.es or Sergi.Balari@uab.es
%% and that of ACL 08 by Joakim Nivre and Noah Smith

\documentclass[11pt]{article}
\usepackage{acl2015}
\usepackage{times}
\usepackage{url}
\usepackage{latexsym}

\setlength\titlebox{5cm}

% You can expand the title box if you need extra space
% to show all the authors. Please do not make the title box
% smaller than 5cm (the original size); we will check this
% in the camera-ready version and ask you to change it back.


\title{Word and sentence similarity\\Project Proposal for NLP Course, Winter 2024}
% https://www.overleaf.com/project/63440c16442dba647b52a1ae
\author{Salveen Singh Dutt \\
  {\tt email@pw.edu.pl} \\\And
  Karina Tiurina \\
  {\tt 01191379@pw.edu.pl} \\ \And 
  Patryk Prusak \\
  {\tt email@pw.edu.pl} \\ \\ 
  supervisor: Anna Wróblewska\\
  Warsaw University of Technology \\
  {\tt anna.wroblewska1@pw.edu.pl}}

\date{}

\begin{document}
\maketitle
\begin{abstract}
  The aim of this project is to address one of a key challenges in natural language processing: the difficulty of accurately assessing sentence similarity across diverse language structures and specialized domains, such as technical support, healthcare, and legal contexts. Existing models often struggle with subtle semantic distinctions and variations in sentence structure, which are critical in real-world applications such as distinguishing between minor delays and significant issues in customer support settings. In this study we are analyzing current sentence similarity models, identifying limitations in domain adaptation, semantic nuance handling, and generalization. Through comprehensive evaluation across multiple benchmark datasets and real-world use cases, our research seeks to provide insights and potential improvements to enhance model performance and reliability.
\end{abstract}

\section{Introduction}
The scientific goal of this project is to enhance the accuracy and reliability of sentence similarity models, which play a critical role in understanding and processing human language in natural language processing (NLP) applications. Sentence similarity, the task of determining how closely two sentences resemble each other in meaning, is fundamental to various applications, such as search engines, customer support systems, and recommendation algorithms. However, current models often face challenges in accurately handling nuanced semantic differences, domain-specific language, and structural variations in sentences, which limits their effectiveness in practical, real-world scenarios.

\subsection{Research Questions}
We aim to explore the following questions:
\begin{enumerate}
    \item How well do existing sentence similarity models handle semantic nuances and domain-specific language across various contexts (e.g., customer support, healthcare, legal)?
    \item What are the primary limitations of current models in generalizing across different domains, and how do these affect practical performance?
    \item What improvements or adaptations could make sentence similarity models more robust in applications that require precise language understanding?
\end{enumerate}

\subsection{Hypotheses}
Our assumptions are:
\begin{enumerate}
    \item Current models may exhibit significant performance drops when applied to specialized domains due to inadequate adaptation to specific terminologies and context-specific meanings.
    \item Addressing these limitations through targeted evaluations can reveal opportunities for improving model performance, providing insights for more robust sentence similarity solutions. 
\end{enumerate}

During this research we aim to investigate these questions and hypotheses, contributing to the development of more accurate, adaptable, and reliable sentence similarity models for NLP applications.

% include your bib file like this:
%\bibliographystyle{acl}
%\bibliography{acl2015}

\end{document}